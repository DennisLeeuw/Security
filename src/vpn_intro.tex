VPN\index{VPN} staat voor Virtual Private Network\index{Virtual Private Network}. Het is een techniek waarmee je \'e\'en of meerdere systemen veilig kan verbinden met \'e\'en of meerdere andere systemen over een publiek netwerk, zoals het Internet. Het maakt het mogelijk om thuis te werken en toch een verbinding met kantoor te hebben alsof je op kantoor bent. Het kan ook twee kantoren met elkaar koppelen over het Internet terwijl de beveiliging zo goed is als was het een priv\'e verbinding over en leased line.

In de beschrijving van VPN is het woord 'veilig' opgenomen. Strict genomen is dit niet waar. Elke vorm van het versturen van lokaal verkeer over een publiek netwerk waarbij informatie herverpakt wordt in een IP packet is een vorm van virtual networking. Op die manier zou ook VLAN tot de mogelijke technieken behoren, maar dat is niet waar we het in dit document over willen hebben.

Om twee kantoren aan elkaar te koppelen werden vroeger (data)lijnen gehuurd van een telecom provider deze zogenaamde leased (gehuurde) lijnen zorgden voor een directe (\'e\'en-op-\'e\'en) koppeling. Omdat de verbinding alleen over het netwerk van de telecom provider ging werd hij als veilig beschouwd.

Om in te kunnen loggen op het bedrijfsnetwerk en thuiswerken mogelijk te maken belde je vroeger in op het bedrijfsnetwerk. Dit gebeurde met modems (tot 33.6 kbps) of met ISDN (64 tot 128 kbps). Het was een veilige verbinding, maar vaak wel een dure oplossing omdat je per minuut gebruik moest betalen.

De opkomst van het Internet zorgde ervoor dat we altijd en overal een netwerk verbinding hebben. Helaas is het Internet per definitie onveilig omdat je over lijnen en netwerken van andere oragnisaties gaat. Je weet nooit wie waar zit mee te luisteren met je verkeer. Het voordeel is dat de netwerkverbinding er als is, dus er hoeven geen leased lijnen meer gehuurd of aangelegd te worden en de thuisgebruiker kan vrijwel onbeperkt gebruikt maken van Internet tegen een vasttarief. De laatste drempel die opgelost moet worden is de veiligheid van de data en dat kan natuurlijk prima met encryptie en authenticatie. Uiteindelijk draait het allemaal om CIA: confidentiality, integrity en availability.
