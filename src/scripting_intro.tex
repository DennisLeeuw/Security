Om een programma te kunnen uitvoeren op een computer moet het programma eerst geschreven worden door een programmeur. De programmeur schrijft programma in een programmeertaal en om er voor te zorgen dat de computer begrijpt wat de programmeur bedoelt moet het programma omgezet worden in machinetaal. Bij een scriptingtaal wordt dat wat de programmeur geschreven heeft on-the-fly omgezet in machinetaal. Dus zodra je het programma aanspreekt wordt het vertaald naar iets wat de CPU snapt, om dit te kunnen doen heb je iets nodig dat het script omzet in machinetaal en dat is de interpreter.

Scripts met data op websites te laten werken zijn er twee soorten scriptingtalen. Je hebt scriptingtalen die op de server staan. Deze talen verwerken eerst de data en maken van de data een webpagina en deze sturen ze op naar de browser van de gebruiker. Je hebt ook talen die aan de client-kant (browser) werken en die in de browser zijn taken uitvoert. Op basis van deze verschil spreken we van server-side of client-side scripting.

Voorbeelden van server-side scripting zijn: PHP, C\#, Perl, Python, Java.

Voor client-side is de scripting taal meestal JavaScript.

Voor de veiligheid maakt het veel uit of een taal op een server draait in een door een systeembeheerder gecontroleerde omgeving of in een browser van een gebruiker.
