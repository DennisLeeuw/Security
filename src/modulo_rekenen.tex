Voor het rekenwerk binnen de crypografie hebben we soms modulo rekenen\index{Modulo rekenen} nodig, ofwel klok rekenen\index{Klok rekenen}. Modulo rekenen wordt wel klok rekenen genoemd omdat dat de meest bekende vorm is die iedereen bijna dagelijks uitvoert. Als iemand tegen je zegt we spreken om 13:00 uur af dan weet iedereen dat die persoon 1 uur 's middags bedoelt. We hebben in gedachte even snel 12 van 13 afgetrokken en hielden 1 over. Modulo rekenen is dus een rekenvorm waarbij we de rest waarde berekenen op basis van een grondgetal. Dat grondgetal bij het klokrekenen is 12. We zouden wiskundig kunnen schrijven 13 mod 12 = 1. Onze klok gaat niet verder dan 24 uur en dus ik klok rekenen een heel beperkte versie van modulo rekenen.

Als we een klok zouden nemen die iets verder doorloopt dan zouden we kunnen zeggen we spreken af om 25:00 uur. 25 mod 12 is weer 1, want we kunnen 2x12 aftrekken van 25 en dan blijven we met een rest 1 achter wat dus weer 1 uur is, alleen nu wel midden in de nacht. Ook de basis kunnen we natuurlijk veranderen: 25 mod 5 is 0, want 5x5 is 25 en 25-25 = 0. We hebben dan dus geen rest getal. Of 38 mod 3 is 2, want 12x3 is 36 en 38-36 = 2.

In programmeertalen en op rekenmachines wordt modulo rekenen vaak weergegeven met een \%. We schrijven dan 25\%12=1, of 25\%5=0.
