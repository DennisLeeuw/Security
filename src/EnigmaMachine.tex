De Engima\index{Enigma} machine is populair geworden door zijn gebruikt door de Nazi's en het feit dat de geallieerden de code konden breken en de berichten mee konden lezen. Toch is de techniek al ouder dan de tweede wereldoorlog.

De Enigma gebruikt een rotor mechanisme om de 26 letter van het alfabet door elkaar te gooien. Bij het intoetsten van een letter op het toetsenbord lichte een letter op op het scherm boven het toetsenbord. De oplichtende letter was de encrypte variant. Dus voor elke toets aanslag moest de oplichtende letter opgeschreven worden, wat vaak door een tweede persoon gedaan werd. Als de encrypte tekst werd ingetoetst dan verscheen op het scherm de bijbehorende gedecodeerde letter, zo kon het oorspronkelijke bericht weer zichtbaar gemaakt worden. Het rotor mechanisme veranderde de verbindingen tussen de toetsten en de lampjes na elke toetsaanslag, dus elke letter werd unique gecodeerd.

Om te zorgen dat de machine niet elke keer alles op dezelfde manier codeert moet hij (dagelijks) worden voorzien van een nieuwe key, die beide partijen moeten kennen, dus een vorm van de one-time-pad. De lijsten met codes moeten dus van te voren bij de verzendende en ontvangende partij bekend zijn en beide partijen moeten op hetzelde moment van key wijzigen.

Het oorspronkelijke ontwerp werd al in 1932 gekraakt door een Poolse wiskundige: Marian Rejewski. Ook de verbeteringen die de Duitsers aanbrachten gedurende de oorlog werden steeds opnieuw gebroken, zodat de Polen en later de geallieerden met Poolse nagemaakte Enigma's de meeste berichten van de Duisters konden blijven volgen.
