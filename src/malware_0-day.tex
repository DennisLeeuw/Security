Malware maakt vaak gebruik van zwakheden in een operating system. Elk stukje software op een computer wordt geschreven door een programmeur en mensen maken fouten. Het zijn geen fouten die het programma verhinderen om te draaien, of die voor de gebruiker merkbaar zijn, maar fouten die bijvoorbeeld te maken hebben met geheugenmanagement. Omdat de fouten niet de werking van het programma in de weg zitten zijn deze fouten vaak lastig om op te sporen. Het programma lijkt goed te functioneren en toch kunnen criminelen gebruik maken van deze fouten om het programma ook andere taken te laten uitvoeren.

Soms zijn security-analisten de criminelen voor en wordt er een security-update van de software uitgebracht voordat de criminelen het foutje vinden. Het analyseren van deze updates levert de criminelen dan weer veel informatie op over het foutje in de software zodat ze alsnog instaat zijn om systemen aan te vallen die geen updates hebben doorgevoerd. Het is daarom noodzakelijk om security-updates zo snel mogelijk door te voeren op systemen.

Het kan ook gebeuren dat criminelen een foutje ontdekken en gebruiken om in te breken op systemen voordat de de rest van de wereld van deze zwakheid weet. In zo'n geval spreken we van een 0-day exploit. Er zijn dus 0 dagen aan beveiliging voor deze exploit. De makers van de software moeten hard aan de slag om een oplossing te verzinnen en deze zo snel mogelijk vrijgeven om de gevolgen zoveel mogelijk te beperken.

