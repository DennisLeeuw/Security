Als je op een website een formulier hebt dat er ongeveer zo uit ziet:
\begin{lstlisting}
<form action="myform.php" method="post">
 <input type="text" name="textfield" value="">
 <input type="submit" name="submit" value="Submit">
</form>
\end{lstlisting}

En een aanvaller vult in het textfield geen gewone tekst in maak bijvoorbeeld een stukje JavaScript:
\begin{lstlisting}
<script>alert("Hello world!")</script>
\end{lstlisting}

dan wordt de JavaScript code uitgevoerd door de browser als je de data zonder verdere controle afbeeld via echo:
\begin{lstlisting}
echo $_POST["textfield"];
\end{lstlisting}

Dit kan je tegen gaan door bijvoorbeeld JavaScript alleen toe te staan als het script geladen wordt als extern script van het eigen domein\index{script-src}, inline JavaScript is dan automatisch verboden:
\begin{lstlisting}
script-src 'self'
\end{lstlisting}
