Als we dezelfde zin versleutelen met een passphrase\index{Passphrase} wordt de encodering nog een stukje complexer. Een passphrase is hetzelfde als een wachtwoord, maar hoeft deze niet perse een woord te zijn, maar kan het ook een hele zin zijn. In de matrix \ref{tab:matrix_passphrase} gebruiken we 'bosje' als passphrase. We schrijven eerst de passphrase in de tabel, deze bepaalt daardoor hoe breed onze tabel wordt. Onder de passphrase zetten we getallen en die verwijzen welke letter het eerst in het alfabet voorkomt. In ons voorbeeld is de 'b' letter die het meest vooraan in het alfabet staat, dus die krijgt waarde \textbf{1} de 'e' is de erop volgende letter en die krijgt dus waarde \textbf{2} en zo voort. De dikgedrukte nummering bepaalt nu de volgorde waarin we de kolom uitlezen.
\begin{table}[h]
\centering
\begin{tabular}{|c|c|c|c|c|}
\hline
	b &
	o &
	s &
	j &
	e \\
\hline
	\textbf{1} &
	\textbf{4} &
	\textbf{5} &
	\textbf{3} &
	\textbf{2} \\

\hline
\hline
	d &
	e &
	z &
	e &
	z \\
\hline
	i &
	n &
	w &
	o &
	r \\
\hline
	d &
	t &
	v &
	e &
	r \\
\hline
	s &
	l &
	e &
	u &
	t \\
\hline
	e &
	l &
	d &
	f &
	g \\
\hline
\end{tabular}
\caption{Bericht in een matrix}
\label{tab:matrix_passphrase}
\end{table}
De cijfertekst wordt dan: didse zrrtg eoeuf entll zwved. De passphrase bepaalt op deze manier hoe de kolommen uitgelezen worden.
