\index{Cryptografie}Cryptografies betekent het schrijven (grafie) in geheim (crypto) schrift. De bedoeling is dat een tekst omgezet wordt in iets dat niet meer leesbaar is. Van klare tekst\index{Klare tekst} (leesbaar) naar een cijfertekst\index{Cijfertekst} (onleesbaar). De omzetting van klare tekst naar cijfertekst heet encryptie\index{Encryptie}. Je hebt natuurlijk niets aan een cijfertekst als je die niet weer leesbaar kan maken. Het omzetten van cijfertekst naar klare tekst heet decryptie\index{Decryptie}.

Er zijn verschillende methodes om klare tekst om te zetten naar cijfertekst. Vroeger werden de omzettingen met pen en papier gedaan de methodes die we daarvoor kennen noemen we daarom handcijfers\index{Handcijfer}, de moderne technieken gebruiken computers en wiskundige methodes en heten algoritmes\index{Algoritme}.

Binnen de wetenschap is de cryptografie een onderdeel van de wiskunde. In de cryptografie houdt men zich bezich met het beveiligen van informatie door het om te zetten in een vorm die alleen door de beoogde ontvanger(s) weer verwerkt kan worden tot bruikbare informatie. Het het kraken van bestaande encryptie-methoden behoort ook tot de cryptografie.

Om data veilig met elkaar te delen zijn er een aantal zaken belangrijk:
\begin{description}
\item[Integer] Een derde persoon kan de berichten niet veranderen
\item[Vertrouwelijk] Een derde persoon kan de berichten niet lezen (encryptie)
\item[Authentiek] Een derde persoon kan zich niet voordoen als de zender of ontvanger
\end{description}
