De moderne browsers, Firefox, Chrome, Edge, Opera, etc. hebben een functie om ontwikkelaars, maar ook nieuwsgierigen, te laten weten wat ze achter de schermen allemaal gedaan hebben om de pagina op het scherm te krijgen. Veel van deze informatie is alleen handig voor website ontwikkelaars maar sommige informatie is ook handig voor het controleren van de veiligheid van websites. Bijna alle browsers gebruiken de F12-toets om deze extra functie op te roepen. Binnen Edge wordt het de Developer Tools genoemd, bij Firefox vind je het in het menu onder Web Developer en Chrome noemt het natuurlijk ook Developer tools, want Edge gebruikt Google Chrome als engine.
