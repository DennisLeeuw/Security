Malware is een samentrekking van Malicious (kwaadaardig) en Software. Het is een stuk software dat vaak niet of moeilijk detecteerbaar door de gebruiker draait op een operating system en waarvan de gebruiker niet weet dat het er is.

Malware wordt vaak \textquote{per ongeluk} ge\"installeerd. Er zijn veel aanvalsvectoren die gebruikt worden om malware op een computer te krijgen. De meest simpele vorm is dat een stukje extra code meekomt met wat lijkt op legale software. Versies van software waarvoor niet betaald hoeft te worden kunnen voorzien zijn malware.

Een andere techniek is via websites. Op een website kan verborgen (javascript) code zitten die een stukje malware installeert op de computer die de website bezoekt.

Een veel gebruikte techniek is via e-mail, waarbij er in de e-mail een executable zit die zich bijvoorbeeld voordoet als Word-document of als PDF, of waarin er een macro wordt gebruikt in een Word of Excel document. Ook een verwijzing naar een website met kwaadaardige code komt vaak voor.

Malware kent verschillende technieken van verspreiding en verschillende functionaliteiten. In dit document zullen we de meest voorkomende behandelen. Moderne malware maakt vaak gebruik van een combinatie van technieken om hun kans van overleven te vergroten.

