Een andere eenvoudige manier om een bericht te versleutelen is het gebruik van een matrix. In de tabel \ref{tab:matrix} hebben we de zin 'Deze zin wordt versleuteld' uitgeschreven in een matrix van 5x5.
\begin{table}[h]
\centering
\begin{tabular}{|c|c|c|c|c|}
\hline
	D &
	e &
	z &
	e &
	z \\
\hline
	i &
	n &
	w &
	o &
	r \\
\hline
	d &
	t &
	v &
	e &
	r \\
\hline
	s &
	l &
	e &
	u &
	t \\
\hline
	e &
	l &
	d &
	0 &
	0 \\
\hline
\end{tabular}
\caption{Bericht in een matrix}
\label{tab:matrix}
\end{table}
Door de tekst op te schrijven van boven naar beneden krijgen we de cijfertekst: didse entll zwved eoeu0 zrrt0. Degene die deze geheime boodschap onderschept moet het formaat van de matrix weten om uit te vinden wat het bericht is. In het voorbeeld hebben we het geencrypte bericht uitgeschreven in blokken van 5, dat is natuurlijk niet zo handig en ook de twee 0'en verraden al veel, maar als we de 0'en vervangen door willekeurige letters en er 1 lange reeks letters van maken dan wordt het al weer een stuk moeilijker om de matrix te raden.

