Het Web of Trust\index{Web of Trust} (WoT\index{WoT}) is ooit bedacht door Phil Zimmermann in 1992 toen hij bezig was met PGP\index{PGP}, Pretty Good Privacy\index{Pretty Good Privacy}. Dit is gebaseerd op onderling vertrouwen zoals in een vriendenkring. Als Bob Alice vertrouwt en Alice vertrouwt Carlos dan zou Bob Carlos ook kunnen vertrouwen. Er is dus geen centrale partij die bepaalt of een certificaat of een publieke sleutel te vertrouwen is en er is ook niet \'e\'en web of trust, net zomin als er \'e\'en vriendenkring is. Er kunnen verschillende vriendenkringen wereldwijd zijn met hun eigen collectie van publieke sleutels die ze vertrouwen.

PGP is ooit ontworpen om e-mail veilig te maken. De gedachte voor de Web of Trust is dat als je public keys met je vrienden uitwisseld je een database kan opbouwen in je e-mail client en als jij aangeeft dat je deze public keys daadwerkelijk van je vrienden hebt gekregen dan ben jij degene die kan zeggen dat ze echt zijn. Als anderen dat ook doen, dan kan je daarna fysiek of electronisch de verzamelde sleutels met elkaar uitwisselen. Daarbij zou het handig zijn als je een gradatie van vertrouwen zou kunnen aangeven. Bob vertrouwt Alice, dus die sleutels zijn volledig te vertrouwen, maar Bob kent Carlos niet, dus het vertrouwen in de sleutels van Carlos zouden een iets lagere vertrouwens waarde kunnen krijgen dan die van Alice.

Het web of trust zoals dat ge\"implementeerd is in PGP en GPG\index{GPG} (GNU Privacy Guard\index{GNU Privacy Guard}, de open source implementatie van de OpenPGP\index{OpenPGP} standaard, heeft de mogelijkheid om ook gebruik te maken van een CA. Binnen een organistatie kan een eigen CA opgezet worden die de certificaten die de public keys bevatten kan certificeren. Op deze manier kan door het vertrouwen van \'e\'en certificaat alle certificaten binnen een web of trust vertrouwd worden.
