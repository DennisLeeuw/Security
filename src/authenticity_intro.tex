Als we communiceren over het Internet hoe weten we dan dat bijvoorbeeld de website van de bank daadwerkelijk de website van de bank is en niet een website opgezet door een boef? Of hoe weten we zeker dat de e-mail die we gekregen hebben daadwerkelijk verzonden is door degene die in het afzenderveld staat? Al deze vragen hebben te maken met authenticiteit\index{Authenticiteit} of Authenticity\index{Authenticity} in het Engels. In het niet digitale leven zou je iemand om zijn paspoort of rijbewijs kunnen vragen zodat de persoon kan aantonen wie die is. In de digitale wereld gebruiken servers certificaten om aan te tonen dat zij echt de server van de bank zijn. Een certificaat is dus te vergelijken met een digitaal paspoort.
