Zoals al eerder aangegeven kan je ook met een private key berichten encrypten en deze de wereld in sturen, iedereen met de public key kan deze berichten decrypten, maar omdat iedereen bij de publiek kan is dit geen vorm van beveiliging, het is wel een vorm van authenticatie, Het is een vorm die we signing\index{signing} noemen. Een ieder die de public key heeft kan het bericht decrypten en weet op dat moment 100\% zeker dat het bericht encrypt is door de degene met de private key. Het is dus een bewijs dat alleen degene die de private key heeft het bericht heeft kunnen versturen. Hiermee kan dit gebruikt worden om bijvoorbeeld contracten te ondertekenen.

Ook bij signing is het natuurlijk weer van belang om de publieke sleutel te kunnen vertrouwen, dus ook hier hebben we weer een certificaat nodig van een CA of een web of trust.
