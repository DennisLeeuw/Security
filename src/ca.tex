Voordat jij vertrouwd wordt om op Schiphol in een vliegtuig te stappen moet je eerst je paspoort laten zien. In je paspoort zit een foto en een beschrijving van een aantal lichamelijke kenmerken waardoor de douane beambte kan controleren dat jij en je paspoort bij elkaar horen. Het feit dat de douane beambte het paspoort vertrouwd heeft te maken met een aantal zaken. Allereerst zal hij of zij het controleren op echtheidskenmerken, zoals watermerken etc. Als dat echt blijkt te zijn, dan is de onderliggende aanname dat het document is uitgeven door de overheid en dat de overheid voor uitgifte een aantal controles heeft gedaan zodat de overheid weet dat het paspoort ook echt bij jou hoort. Het is dus de overheid die ervoor zorgt dat jij alleen jouw paspoort hebt en ook alleen jij het paspoort hebt. Tevens is het paspoort maar beperkt geldig.

Al deze dingen koment we ook tegen bij het digitale paspoort, het certificaat, dat door de Certificate Authority\index{Certificate Authority} (CA\index{CA}) wordt uitgegeven. Als je een certificaat nodig hebt, dan zul je dat moeten aanvragen. Om dit te kunnen doen moet je eerst een public en private key paar genereren. Op basis van de public key maak je een Certificate Signing Request\index{Certificate Signing request} CSR\index{CSR}. In dit verzoek tot het tekenen van het certificaat neem je een aantal zaken op zoals bijvoorbeeld waar het certificaat voor bedoeld is. Als het voor een webserver is dan neem je de domeinnaam op in het CSR, de geldigheidsduur en de eigenaar van de website. Dit verzoek stuur je op naar een CA. De CA controleert de gegevens. Dus die kijkt na of jij echt de eigenaar bent van het domein en of jij echt bent wie je bent. Als alles klopt dan krijg je van de CA een certificaat met de gegevens uit de CSR, maar nu getekend met de handtekening van de CA. De CA ondertekend het document met zijn private key. Hiermee heb je het digitale paspoort waarmee je kan aantonen dat jouw server de echte server is die bij je domeinnaam hoort.


