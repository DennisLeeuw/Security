Een worm lijkt erg op een virus, maar er zijn enkele verschillen. Een worm verspreidt zich via het netwerk en maakt gebruik van zwakheden in de netwerk-diensten of de netwerk-stack van een systeem. Een ander verschil met een virus is dat een worm zichzelf kan opstarten nadat het een systeem ge\"infecteerd heeft. Waar een virus altijd de mens nodig heeft voor activatie heeft een worm \'e\'enmalig de mens nodig om op te starten daarna zoekt hij zelf zijn weg.

Wormen maken gebruik van zwakheden in de netwerk faciliteiten van een besturingssysteem. Het is dus geheel afhankelijk van welk proces er misbruikt wordt met welke rechten een worm kan draaien op een systeem. In het slechtste scenario draait de worm met rechten op kernel-niveau.

