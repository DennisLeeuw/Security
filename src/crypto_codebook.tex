\index{Codebook}\index{Substitutie!Codeboek}In de codeboek versleuteling gaat het om de omzetting van hele stukken tekst door een andere. Er worden dan zinsdelen omgezet door gebruik te maken van dezelfde omzettingstabel. Stel dat een generaal een bericht over wil brengen over hoe troepen zich moeten verplaatsen. Bijvoorbeeld: "Infanterie devisie 304 naar het noorden." Als we afspreken de "infanterie divisie 304" aangeduid wordt als "de olifant" en het noorden als "de oever", dan zou de tekst kunnen worden "De olifant naar de oever". Voor de vijand zegt dit niets als ze niet over dezelfde codebook beschikken. De hoeveelheid combinaties is theoretisch eindeloos, maar het is nog steeds zo dat de substitutie vast is, dus altijd zal "infanterie divisie 304" vertaald worden met "de olifant", dus als je maar genoeg berichten onderschept en voldoende troepen bewegingen waarneemt dan zal je uiteindelijk ontdekken welk zinsdeel wat betekent.
