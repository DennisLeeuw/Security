PKI\index{PKI} staat voor Public Key Infrastructure\index{Public Key Infrastructure} en is het totaal van maatregelen en instanties die nodig zijn om publieke sleutels te beheren. PKI is bedoelt om gebruikers en apparaten te authenticeren in een digitale wereld, vergelijkbaar met het paspoort in de gewone wereld. Het gaat daarbij om vertrouwen, vertrouwen in de echtheid van bijvoorbeeld een website die je bezoekt, de echtheid van een document, of het vertrouwen dat een gebruiker die gebruik wil maken van een systeem ook werkelijk die gebruiker is.

Vertrouwen tussen digitale entiteiten wordt bepaald door het vertrouwen dat we hebben in de public key van een entiteit. De private key moet altijd geheim blijven, maar de publieke sleutel mogen we met iedereen delen. De vraag bij de verkregen publieke sleutel is altijd of dit de sleutel is van de machine of persoon waarmee ik contact wil hebben of dat er ergens een man-in-the-middle zit die mij een valse sleutel heeft gegeven en dus stiekem met ons meeleest.

Er zijn verschillende oplossing om dit vertrouwen te bewerkstelligen:
\begin{description}
\item[CA] Certificate Authority
\item[WoT] Web of Trust
\end{description}
