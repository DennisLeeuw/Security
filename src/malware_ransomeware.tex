Een andere naam voor ransomeware is de vaak in de media gebruikte term cryptolocker. Ransomeware locked een computer en/of de bestanden om daarna de gebruiker te vragen om te betalen zodat de bestanden of de computer weer geunlocked kunnen worden. Het is niet gegarandeerd dat het betalen ook daadwerkelijk zorgt dat de bestanden weer toegankelijk worden, het gebeurt vaak dat er wel betaald wordt, maar dat de sleutel tot het unlocken niet geleverd wordt.

Cryptolockers maken gebruik van de public key infrastructure. De public key wordt gebruikt om de bestanden te encrypten. Na betaling hoop je de private key te krijgen om alle bestanden te decrypten. De rechten van de gebruiker waaronder de cryptolocker draait bepaalt waar de cryptolocker bij kan en welke bestanden er versleuteld kunnen worden. Alleen de bestanden waar de gebruiker schrijfrechten op heeft kunnen door de cryptolocker ge-encrypt worden.

