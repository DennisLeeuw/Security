Als je wilt inloggen op een website dan zal je je gebruikersnaam en wachtwoord moeten opgeven, bij alleen gebruik van HTTP zou die informatie als platte-tekst over het Internet gaan. Als er onderweg iemand het verkeer afluistert (man-in-the-middle) dan zou die zo je inlog-gegevens kunnen achterhalen. Om dit te voorkomen maken we gebruik van HTTPs\index{HTTPs}, kortom een HTTP-verbinding over een geencrypte verbinding. Voor deze versleutelde verbinding werd voorheen SSL (Secure Socket Layer)\index{SSL}\index{Secure Socket Layer} gebruikt, tegenwoordig heet het TLS\index{SSL} (Transport Layer Security)\index{Transport Layer Security}.
